% A Presentation Program based on LaTeX Beamer
% Lu Niu
% LukeNiu@outlook.com

\documentclass[UTF8,beamer,serif,ctexart]{beamer}
	
\usepackage{ctex}
\usepackage[UTF8,space,hyperref]{}
\usepackage{amsmath,amsfonts,amssymb,bm}
\usepackage{latexsym}
\usepackage{color}
\usepackage{cjk}
\usepackage{graphicx,hyperref,url}
%\usepackage{ctex}								% in Chinese

\usetheme{Warsaw}								% Style

\definecolor{titlebg}{RGB}{040,120,180}

\begin{document}

\mode<presentation>
{
	\setbeamercolor{title}{fg=white,bg=titlebg}	% Main title color
	\setbeamercolor{structure}{fg=titlebg}		% Structure color
	\setbeamerfont{author}{shape=\bfseries,family=\rmfamily}
	\setbeamerfont{title}{shape=\bfseries,family=\sffamily}
	\setbeamerfont{subtitle}{shape=\bfseries,family=\sffamily}
	\setbeamertemplate{footline}[frame number]	% Footline '12/36'
}

\author[Lu Niu]
{
	Lu Niu \\
	\medskip
	{\small \url{LukeNiu@outlook.com}}
}
\institute[Lu Niu]
{
	School of Mathematics and Physics \\ 
	University of Science and Technology Beijing
}
\date{December 20, 2017}

\title{中文ChinsesThe Concept of Qubit}
\subtitle{The information unit under the view of Quantum Mechanics}

\setlength{\parindent}{1em}

\begin{frame}%
	\begin{titlepage}
	\end{titlepage}
\end{frame}

\begin{frame}{CONTENT}{}
%	\tableofcontents  					% Make content besed on section
\begin{itemize}  
	\item Introduction
	\item Bit and Qubit
	\item Conceptual Representation
	\item Quantum States
	\item Quantum Entanglements
\end{itemize} 
\end{frame}

%\section{Section A}
\begin{frame}{Introduction}{}
	\par In Quantum Information and Quantum Computation, Qubit is the unit of information, the quantum analogue of the classical binary bit. 
	\par A qubit is a two-state quantum-mechanical system, such as the polarization of a single photon: here the two states are vertical polarization and horizontal polarization.
	\par In a classical system, a bit would have to be in one state or the other. However, quantum mechanics allows the qubit to be in a superposition of both states at the same time, a property that is fundamental to quantum computing.
	\par In other words, the difference between Bit and Qubit is that a Qubit can be in a state other than 0 or 1. It is also possible to form linear combinations of states, often called superpositions:
	\begin{equation}
		\left| \psi \right\rangle = a\left| 0 \right\rangle+b\left| 1 \right\rangle
	\end {equation}
	\par \setlength{\parindent}{1em} The numbers $a$ and $b$ are complex numbers. The state of a qubit is a vector in a two-dimensional complex vector space.
\end{frame}

\begin{frame}{Bit and Qubit}{}
	\begin{itemize}  
	\item Bit:
	\end{itemize} 
	\par The bit is the basic unit of information. It is used to represent information by computers. Regardless of its physical realization, a bit has two possible states typically thought of as 0 and 1, but more generally—and according to applications—interpretable as false and true respectively, or any other dichotomous choice. 
	\begin{itemize}  
	\item Qubit:
	\end{itemize} 
	\par A qubit has a few similarities to a classical bit, but is overall very different. There are two possible outcomes for the measurement of a qubit—usually 0 and 1, like a bit. The difference is that whereas the state of a bit is either 0 or 1, the state of a qubit can also be a superposition of both.
\end{frame}

\begin{frame}{Conceptual Representation}{}
	\par The two states in which a qubit may be measured are known as basis states (or basis vectors):
	\begin{equation}
		\left\{
		\begin{aligned}
			0 & = \begin{bmatrix} 1 \\ 0 \end{bmatrix} \\
			1 & = \begin{bmatrix} 0 \\ 1 \end{bmatrix} 
		\end{aligned}
		\right.
	\end {equation}
	\par Qubits can also be compound, the compound qubits can be described:
	\begin{equation}
			00 = \begin{bmatrix} 1 \\ 0 \\ 0 \\ 0 \end{bmatrix} ,
			01 = \begin{bmatrix} 0 \\ 1 \\ 0 \\ 0 \end{bmatrix} ,
			10 = \begin{bmatrix} 0 \\ 0 \\ 1 \\ 0 \end{bmatrix} ,
			11 = \begin{bmatrix} 0 \\ 0 \\ 0 \\ 1 \end{bmatrix} 
	\end {equation}	
\end{frame}

\begin{frame}{Quantum States}{}
	\par As shown as Equation.1, a pure qubit state is a linear superposition of the basis states. This means that the qubit can be represented as a linear combination of $\left| 0 \right\rangle$ and $\left| 1 \right\rangle$. Where a and b are probability amplitudes and can in general both be complex numbers.
	\par When we measure this qubit in the standard basis, the probability of outcome $\left| 0 \right\rangle$ is $|a|^{2}$, and the probability of outcome $\left| 1 \right\rangle$ is $|b|^{2}$. Because the absolute squares of the amplitudes equatie of probabilities, it follows that $a$ and $b$ must be constrained by the equation: 
	\begin{equation}
		|a|^{2}+|b|^{2}=1
	\end {equation}	
\end{frame}

\begin{frame}{Quantum States}{Bloch Sphere}
\par It might, seem that there should be four degrees of freedom, as $a$ and $b$ are complex numbers with two degrees of freedom each. However, one degree of freedom is removed by the normalization constraint, , which can be treated as the equation for a 3-sphere embedded in 4-dimensional space with a radius of 1 (unit sphere). This means, with a suitable change of coordinates, one can eliminate one of the degrees of freedom. One possible choice is that of Hopf coordinates:
\begin{equation}
\left\{
\begin{aligned}
a & = e^{i\psi}\cos\frac{\Theta}{2},	\\
b & = e^{i\psi+\phi}\cos\frac{\Theta}{2},
\end{aligned}
\right.
\end {equation}	
\end{frame}

\begin{frame}{Quantum States}{Bloch Sphere}
	\par Additionally, for a single qubit the overall phase of the state $e^{i\psi}$ has no physically observable consequences, so we can arbitrarily choose $a$ to be real (or $b$ in the case that $a$ is zero), leaving just two degrees of freedom:
	\begin{equation}
		\left\{
		\begin{aligned}
			a & = \cos\frac{\Theta}{2},	\\
			b & = e^{i\phi}\cos\frac{\Theta}{2},
		\end{aligned}
		\right.
	\end {equation}	
\end{frame}

\begin{frame}{Quantum States}{Bloch Sphere}
	\par The possible states for a single qubit can be visualised using a Bloch sphere (see diagram below). Represented on such a sphere, a classical bit could only be at the North Pole or the South Pole, in the locations where $\left| 0 \right\rangle$ and$\left| 1 \right\rangle$.
	\par The rest of the surface of the sphere is inaccessible to a classical bit, but a pure qubit state can be represented by any point on the surface. 
	\par \includegraphics[height=100px]{Sphere.png}
	\par Question: where is the qubit state $\dfrac{1}{\sqrt{2}}(\left|0\right\rangle+i\left|1\right\rangle)$?
\end{frame}

\begin{frame}{Quantum States}{Operations on Pure Qubit States}
	\par There are various kinds of physical operations that can be performed on pure qubit states.
	\par 1. A quantum logic gate can operate on a qubit: mathematically speaking, the qubit undergoes a unitary transformation. Unitary transformations correspond to rotations of the qubit vector in the Bloch sphere.
	\par 2. Standard basis measurement is an operation in which information is gained about the state of the qubit. The result of the measurement will be a pure qubit state.  Measurement of the state of the qubit alters the values of $a$ and $b$. For example, if the result of the measurement is $\left|0\right\rangle$, $a$ is changed to 1 and $b$ is changed to 0. Note that a measurement of a qubit state entangled with another quantum system transforms a pure state into a mixed state.
\end{frame}

\begin{frame}{Quantum States}{Mixed State}
	\par It is possible to put the qubit in a mixed state, a statistical combination of different pure states. Mixed states can be represented by points inside the Bloch sphere. A mixed qubit state has three degrees of freedom: the angles $\phi$ and $\theta$ , as well as the length $r$ of the vector that represents the mixed state.
\end{frame}

\begin{frame}{Quantum Entanglement}{}
	\par An important distinguishing feature between a qubit and a classical bit is that multiple qubits can exhibit quantum entanglement. Entanglement is a nonlocal property that allows a set of qubits to express higher correlation than is possible in classical systems. Take, for example, two entangled qubits in the Bell State:
	\begin{equation}
		\dfrac{1}{\sqrt{2}}(\left|00\right\rangle+\left|11\right\rangle)
	\end {equation}	
\end{frame}

\begin{frame}{Quantum Entanglement}{Bell State}
	\par The Bell states are four specific maximally entangled quantum states of two qubits.
	\par The qubits are usually thought to be spatially separated. Nevertheless, they exhibit perfect correlation, which cannot be explained without quantum mechanics. the form of convenient basis of the two-qubit Hilbert space is that:
\begin{equation}
	\left\{
		\begin{aligned}
			\left|\Phi^{+}\right\rangle & = \dfrac{1}{\sqrt{2}}(\left|0\right\rangle_{\rm{A}}\otimes\left|0\right\rangle_{\rm{B}}+\left|1\right\rangle_{\rm{A}}\otimes\left|1\right\rangle_{\rm{B}})	\\
			\left|\Phi^{-}\right\rangle & = \dfrac{1}{\sqrt{2}}(\left|0\right\rangle_{\rm{A}}\otimes\left|0\right\rangle_{\rm{B}}-\left|1\right\rangle_{\rm{A}}\otimes\left|1\right\rangle_{\rm{B}})	\\
			\left|\Psi^{+}\right\rangle & = \dfrac{1}{\sqrt{2}}(\left|0\right\rangle_{\rm{A}}\otimes\left|1\right\rangle_{\rm{B}}+\left|1\right\rangle_{\rm{A}}\otimes\left|0\right\rangle_{\rm{B}})	\\
			\left|\Psi^{-}\right\rangle & = \dfrac{1}{\sqrt{2}}(\left|0\right\rangle_{\rm{A}}\otimes\left|1\right\rangle_{\rm{B}}-\left|1\right\rangle_{\rm{A}}\otimes\left|0\right\rangle_{\rm{B}})	\\
		\end{aligned}
	\right.
\end {equation}	
\end{frame}

\begin{frame}{}{}
	\par Thanks
\end{frame}

\end{document}